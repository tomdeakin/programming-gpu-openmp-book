

\chapter{Writing performant and portable code}
\label{chapter:portable}

The book so far as taught general concepts of GPUs and the parts of OpenMP needed to program GPUs.
This chapter puts everything together, with a series of case studies.
This takes things to the next level in writing performant code on GPUs.
We'll discuss specific architectural details of GPUs and how to optimize and write fast code for them.

We'll focus on:
\begin{itemize}
  \item Case studies.
  \item Optimizing for fast memory access.
  \item Demonstrate the parts of OpenMP needed to obtain this good performance.
  \item Combine multiple OpenMP directive shown in the previous chapters to show more complex use cases.
\end{itemize}

\section{Approaches to optimization}
\begin{itemize}
  \item Discuss Wen Mei Hwu's approach to optimization as it is extremely powerful.
\end{itemize}

\subsection{Coalesced memory access patterns}
\subsection{Column-major vs Row-major memory access}
\subsection{Example: Jacobi solver or 5-point stencil}

\subsection{Array-of-Structure vs Structure-of-Arrays}
\subsection{Example: multi-channel image convolution}

\subsection{Team only memory}
\begin{itemize}
  \item Equivalent of OpenCL local memory (CUDA shared memory).
  \item Memory shared between threads in a team.
  \item Uses OpenMP 5.0 allocators: omp\_pteam\_mem\_alloc
\end{itemize}
\subsection{Example: Matrix multiplication}

\section{Performance portability motivation}
\begin{itemize}
  \item Computer architectures changing.
  \item Lots of vendors, CPUs, GPUs etc.
  \item Want to write code that can run well, and equally well, on all platforms.
\end{itemize}


\section{Directive recommendations use target teams distribute parallel for simd}
\begin{itemize}
  \item Already recommended the combined directive in Chapter~\ref{chapter:parallelism}.
  \item Explain the dangers here.
  \item Part of the picture is the hierarchy of parallelism in OpenMP.
\end{itemize}

\subsection{GPU architecture review}
Quickly summarize the model explained in Chapter~\ref{chapter:heterogeneity}.

\subsection{Mapping OpenMP constructs to GPUs}
\begin{itemize}
  \item Three levels of parallelism in OpenMP vs two in hardware.
  \item Pick a specific GPU. Map high level abstraction of GPU device model to different real GPU hardware.
  \item Two levels of parallelism exposed in GPU hardware.
  \item Show there is a choice of mapping teams, threads and SIMD onto two levels.
  \item This is why specifying the number of teams and threads is not portable.
  \item Case study: Older Cray compilers vs LLVM compilers.
  \item Cray used to map teams to compute units, and simd to processing elements. Threads are ignored.
  \item LLVM maps teams to compute units, threads to processing elements. SIMD is ignore.
  \item Modern cray (9 and 10) follow LLVM approach.
\end{itemize}

\section{Case studies of OpenMP doing well}
\begin{itemize}
  \item Some case studies from experience.
  \item Use P3HPC studies.
\end{itemize}

\section{Clause recommendations}
\begin{itemize}
  \item In general, don't specify number of threads, number of teams, schedule, etc.
  \item Some clauses are helpful: collapse.
\end{itemize}

\subsection{collapse}
\begin{itemize}
  \item This was explained in Chapter~\ref{chapter:overview}.
  \item Useful for exposing all the parallelism.
\end{itemize}




\section{Metadirectives}
\begin{itemize}
  \item Want to write one version which runs on CPUs and GPUs.
  \item Metadirectives might help.
  \item Parallelism is universal and compilers ignore the right bits.
  \item Target tasks probably ignored by compiler too.
  \item What about the two memory spaces?
\end{itemize}

