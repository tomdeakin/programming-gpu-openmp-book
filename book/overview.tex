
\chapter{OpenMP overview}
\label{chapter:overview}

Shows everything they might need to review before the start on this book.

\section{Quick crash course in OpenMP}
\begin{itemize}
  \item What OpenMP shared memory is: cores share memory.
  \item Annotate code with compiler directives.
  \item API calls.
\end{itemize}

\subsection{Fork-join parallelism}
\begin{itemize}
  \item Parallel: threads in a team.
  \item Worksharing loops: parallel for.
  \item Clauses to change effects: num\_threads, etc.
  \item schedule(static) important to introduce as will need it later.
  \item collapse() clause.
  \item Keep simple enough so reader knows notation, but refer to Common Core and Using OpenMP book for more details.
\end{itemize}

\subsection{Shared memory between threads}
\subsubsection{Data sharing clauses}
\begin{itemize}
  \item private and firstprivate.
  \item shared and default(none) clauses.
  \item Need to be careful how we explain “shared memory” given the memory sharing rules between host/target we’ll introduce later.
\end{itemize}

\section{Example: serial to parallel vector add with stack arrays}
\begin{itemize}
  \item vector add.
  \item arrays allocated on the stack instead of heap.
  \item no reduction yet.
  \item can reuse this as the first target example without needing map clauses.
  \item stack arrays might also be useful for first target example to highlight data sharing clauses.
\end{itemize}

\section{Example: serial to parallel pi}
\begin{itemize}
  \item Pi reduction example.
  \item Shows use of reduction clause.
\end{itemize}

\section{Show the BUD}
Climax. Here is BUD. Book is going to explain it.




