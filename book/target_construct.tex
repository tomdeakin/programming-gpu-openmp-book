
\chapter{Running on a target device}
\label{chapter:target}

\section{Host/device model}
\begin{itemize}
  \item Host with target attached.
  \item Target has its own memory space.
  \item Can have more than one target, device() clause.
  \item Data moved between host and device by implicit rules and explicit directives.
\end{itemize}

\subsection{Target construct offloading execution}
\begin{itemize}
  \item Target directive.
  \item Defines movement of execution to the target device.
  \item Host waits for the region to finish.
  \item Don't want too much detail on the tasking model here because will go into it in Chapter~\ref{chapter:async}. It's not relevant for getting starting this early on.
\end{itemize}

\section{The target memory environment and implicit mapping rules.}
\begin{itemize}
  \item target directive also triggers data movement as well as execution.
  \item first place where data movement is allowed, and will introduce the others at appropriate points.
  \item Different kinds of data we want to move: scalars, (stack) arrays, heap data/pointers, structures.
\end{itemize}

\subsection{Scalar variables}
\begin{itemize}
  \item Mapped as firstprivate.
  \item Example: {\tt int N; double x;}
  \item a scalar struct, as long as it is a complete type
  \item {\bf Not} copied back to the host at end of target region.
\end{itemize}

\subsection{Stack arrays}
\begin{itemize}
  \item Fixed sized stack arrays.
  \item Example: {\tt double arr[1024];}
  \item Complete types. Arrays of structs if complete type.
  \item Copied to device at start of target region, copied back at the end.
  \item Host not allowed to use copy in the meantime (with further details on this in Chapter~\ref{chapter:async}\dots).
  \item Data shared between {\bf all threads} on a device.
\end{itemize}

\subsection{Warning: heap arrays and pointers - reference later chapter}
\begin{itemize}
  \item We don't introduce the map clause yet.
  \item This is done in Chapter~\ref{chapter:memory}.
  \item This section says that we must do something explicit for everything not covered by the implicit rules.
  \item Examples of those would be heap arrays, and data-structures with pointers.
\end{itemize}

\section{Example: Vector add. arrays on stack. No need for map clauses yet.}
\begin{itemize}
  \item Takes vector add example from Chapter~\ref{chapter:overview}.
  \item Arrays are allocated on the stack, so follows implicit mapping rules.
  \item Example will simply transfer execution.
  \item Parallelism comes in Chapter~\ref{chapter:parallelism}.
  \item In OpenCL, have to deal with the host API copying buffers to/from host and device before we can run a meaningful kernel. Thinking in OpenCL might help here.
\end{itemize}

