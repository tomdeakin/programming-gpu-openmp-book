\begin{figure*}[!tb]
\begin{verbatim}
1 #pragma omp target map(a,b,c,d)
2 {
3   for (i=0; i<N; i++) {
4     a[i] =  b[i] * c + d;
5   }
6 } // End of target
\end{verbatim}
\caption{ \textbf {Code fragment with one target region} -- \small
          The target region is executed by a thread running on
          an accelerator.
         }
\label{figure:chapter6-device-v1}
\end{figure*}


\begin{figure*}[!tbhp]
\begin{verbatim}
1 omp target map(a,b,c,d)
2 {
3   #pragma parallel for
4   for (i=0; i<N; i++) {
5     a[i] =  b[i] * c + d;
6   }
7 } // End of target
\end{verbatim}
\caption{ \textbf {Augmented code fragment with a parallel region} -- \small
          The parallel region is executed by a team of threads running
          on an accelerator.
        }
\label{figure:chapter6-device-v2}
\end{figure*}


\begin{figure*}[!tbh]
\begin{verbatim}
 1 #pragma omp target map(a,b,c,d) nowait // Generate target task
 2 {
 3   #pragma parallel for
 4   for (i=0; i<N; i++) {
 5     a[i] =  b[i] * c + d;
 6   }
 7 } // End of target
 8 
 9 F(b); // Execute in parallel with target task
10
11 #pragma omp taskwait // Wait for target task to finish
\end{verbatim}
\caption{ \textbf {Code fragment with a target nowait region} -- \small
          The encountering thread generates a target task 
          and then continues past the target construct
          to execute the function \emph{F()}.
         }
\label{figure:chapter6-nowait}
\end{figure*}


\begin{figure*}[!tb]
\begin{verbatim}
1   char *hptr = malloc(N);
2
3   // Error - Accessing a host address on accelerator
4   #pragma omp target map(hptr)
5   for (int i=0; i<N; i++)
6     *hptr++ = 0;
\end{verbatim}
\caption{ \textbf {Illegal access of a host memory address } -- \small
          A pointer variable containing a host memory address cannot be
          de-referenced by an accelerator thread.
         }
\label{figure:chapter6-devptr1}
\end{figure*}


\begin{figure*}[!tb]
\begin{verbatim}
1   char *dptr;
2   #pragma omp target map(dptr)
3   dptr = malloc(N);
4
5   // Error - Accessing a device address on host
6   for (int i=0; i<N; i++)
7     *dptr++ = 0;
\end{verbatim}
\caption{ \textbf {Illegal access of an accelerator memory address} -- \small
          A pointer variable containing an accelerator memory address 
          cannot be de-referenced by a host thread.
         }
\label{figure:chapter6-devptr2}
\end{figure*}


\begin{figure*}[!tb]
\begin{verbatim}
1   int dev = omp_get_default_device();
2   char *dptr = omp_target_alloc(dev, n);
3 
4   #pragma omp target is_device_ptr(dptr)
5   for (int i=0; i<n; i++)
6     *dptr++ = 0;
\end{verbatim}
\caption{ \textbf {Legal access of an accelerator memory address using a device pointer} -- \small
          A device pointer variable that appears in an \texttt{is\_device\_ptr} clause 
          may be de-referenced in a target region.
        }
\label{figure:chapter6-devptr3}
\end{figure*}


\begin{figure*}[!tbhp]
\begin{verbatim}
1   char *hptr;
2
3   hptr = malloc(1024);
4
5   // Map an array section.
6   #pragma omp target map(hptr[0:1024])
7   for (int i=0; i<N; i++)
8     hptr[i] = 0;
9   
\end{verbatim}
\caption{ \textbf {Map a pointer-based array section} -- \small
          Use an array section to map pointed-to memory.
         }
\label{figure:chapter6-array-sections1}
\end{figure*}


\begin{figure*}[!tbhp]
\begin{verbatim}
 1 float *p = malloc(N);
 2 float a[N];
 3
 4 // Map pointer based array section
 5 map(p[0:N:1]) 
 6 map(p[0:N])
 7 map(p[:N])
 8
 9 // Map array based array section
10 map(A[0:N:1])
11 map(A[0:N])
12 map(A[:N])
13 map(A[:]) // Size is N
14 
15 // Map array section with offset
16 map(p[32:N-32]
17 map(A[N/2:N/4]
18
\end{verbatim}
\caption{ \textbf {Array section syntax examples} -- \small
          Various usage of array section syntax in C and C++.
         }
\label{figure:chapter6-map-ptr1}
\end{figure*}


\begin{figure*}[!tb]
\begin{verbatim}
 1 #define BIG 256
 2 #define N (1024*1024)
 3 int a[N*BIG];
 4 
 5 void F(const int c, const int d)
 6 {
 7   for (int k=0; k<N*BIG; k+=N) {
 8     #pragma omp target map(from:a[k:N]) firstprivate(c,d) 
 9     for (int i=0; i<N; i++) {
10       a[k+i] =  k+i * (c + d);
11     } // End of target
12   }
13 }
\end{verbatim}
\caption{ \textbf {Use array section to map a subset of an array} -- \small
          Map a slice of the array \texttt{a} each time through the loop.
         }
\label{figure:chapter6-mapslice}
\end{figure*}


\begin{figure*}[!tbhp]
\begin{verbatim}
 1 #include <stdio.h>
 2 #include <omp.h>
 3 void hello(void)
 4 {
 5   #pragma omp target 
 6   {
 7     if (!omp_is_initial_device())
 8       printf("Hello World from accelerator\n");
 9     else
10       printf("Hello World from host\n");
11   }
12 }
\end{verbatim}
\caption{ \textbf {Example of a target construct } -- \small
          If the initial thread is running on an accelerator, it executes the
          first \texttt{printf()}.
          Otherwise, it is running on the host device and
          executes the second \texttt{printf()}.
          Note that some implementations may not support calling \texttt{printf()} on an accelerator. 
         }
\label{figure:chapter6-hello}
\end{figure*}


\begin{figure*}[!tb]
\begin{verbatim}
 1 void mxv(int m, int n, double * restrict a,
 2          double * restrict b, double * restrict c)
 3 {
 4 
 5   #pragma omp target map(a[:n],b[:n],c[:n]) firstprivate(m,n)
 6   {
 7     int i, j;
 8     #pragma omp parallel for default(none) \
 9             shared(m,n,a,b,c) private(i,j)
10     for (i=0; i<m; i++)
11     {
12       a[i] = b[i*n]*c[0];
13       for (j=1; j<n; j++)
14         a[i] += b[i*n+j]*c[j];
15     } // End of parallel for
16   } // End of target
17 }
\end{verbatim}
\caption{ \textbf {Example using the target construct to execute the matrix 
                   times vector on an accelerator} -- \small
          The host thread waits for the execution of the target region to
          finish before it continues after the construct.
         }
\label{figure:chapter6-target-mxv}
\end{figure*}


\begin{figure*}[!tb]
\begin{verbatim}
 1 #include <omp.h>
 2 extern void do_team_work(int, int, int, int);
 3 #pragma omp declare target(do_team_work)
 4 void f()
 5 {
 6   #pragma omp target teams
 7   {
 8     int team = omp_get_team_num();
 9     int nteams = omp_get_num_teams();
10     int tid = omp_get_thread_num(); // Always 0
11     int nthreads = omp_get_num_threads(); // Always 1
12     do_team_work(team, nteams, tid, nthreads);
13   } // End of target teams
14 }
\end{verbatim}
\caption{ \textbf {Example of the target teams construct } -- \small
          Multiple initial threads execute the function \texttt{do\_team\_work()}.     
         }
\label{figure:chapter6-teams-v1}
\end{figure*}


\begin{figure*}[!tbh]
\begin{verbatim}
 1 #include <omp.h>
 2 extern void do_team_work(int, int, int, int);
 3 #pragma omp declare target(do_team_work)
 4 void f()
 5 {
 6   #pragma omp target teams num_teams(4)
 7   #pragma omp parallel num_threads(5)
 8   {
 9     int team = omp_get_team_num();
10     int nteams = omp_get_num_teams();
11     int tid = omp_get_thread_num();
12     int nthreads = omp_get_num_threads();
13     do_team_work(team, nteams, tid, nthreads);
14   } // End of target teams
15 }
\end{verbatim}
\caption{ \textbf {Example of a parallel construct nested in a 
                   target teams construct} -- \small
          Multiple teams of threads execute the function \texttt{do\_team\_work()}.     
         }
\label{figure:chapter6-teams-v2}
\end{figure*}


\begin{figure*}[!tb]
\begin{verbatim}
1 void saxpy(float *restrict y, float *restrict x, float a, int n)
2 {
3   #pragma omp target teams map(y[:n]) map(to:x[:n]) 
4   #pragma omp distribute
5   for (int i=0; i<n; i+=n)
6   {
7       y[i] =  y[i] + a*x[i];
8   }
9 }
\end{verbatim}
\caption{ \textbf {Example of the distribute worksharing construct} -- \small
          Each initial thread created by the target teams construct 
          executes a subset of the iterations in the loop's iteration space.
         }
\label{figure:chapter6-distribute-v1}
\end{figure*}


\begin{figure*}[!tbh]
\begin{verbatim}
 1 void saxpy(float *restrict y, float *restrict x, float a, int n)
 2 {
 3   // Assume n is even
 4   #pragma omp target teams map(y[:n]) map(to:x[:n]) num_teams(2) 
 5   #pragma omp distribute
 6   for (int j=0; j<n; j+=n/2)
 7   {
 8     #pragma omp parallel num_threads(4)
 9     #pragma omp for
10     for (int i=j; i<n/2; i++)
11       y[i] = y[i] + a*x[i];
12   }
13 }
\end{verbatim}
\caption{ \textbf {Example of worksharing a loop across two levels of
                   parallelism} -- \small
          Use team level parallelism on the outer loop and thread level
          parallelism on the inner loop.  Distribute the loop iterations to two
          teams.  Each team then uses four threads to execute the iterations
          that are assigned to it.  
         }
\label{figure:chapter6-distribute-v2}
\end{figure*}


\begin{figure*}[!tbh]
\begin{verbatim}
1 void saxpy(float *restrict y, float *restrict x, float a, int n)
2 {
3   #pragma omp target teams map(y[:n]) map(to:x[:n]) 
4   #pragma omp distribute parallel for
5   for (int i=0; i<n; i++)
6     y[i] = y[i] + a*x[i];
7 }
\end{verbatim}
\caption{ \textbf {Example of the distribute parallel loop accelerated 
                   worksharing construct} -- \small
          Create multiple thread teams executing in parallel.
          Distribute loop iterations to the teams and then
          to the threads in each team.
         }
\label{figure:chapter6-distribute-v3}
\end{figure*}


\begin{figure*}[!tbh]
\begin{verbatim}
 1 int dotp(int *restrict a, int *restrict b, int n)
 2 {
 3   int sum = 0;
 4 
 5   #pragma omp target teams distribute map(to:a[:n],b[:n]) \
 6                                       reduction(+:sum)
 7   for (int i=0; i<n; i++)
 8     sum += a[i] * b[i];
 9 
10   return sum;  // Sum is always 0
11 }
\end{verbatim}
\caption{ \textbf {A variable cannot appear in both map and reduction clauses on the same construct} -- \small
          The \texttt{reduction} clause is associated with the
          \texttt{teams} directive.
          The variable \texttt{sum} is not mapped, and therefore,
          the reduced value of \texttt{sum} is lost after the region.
        }
\label{figure:chapter6-targetteams-reduction}
\end{figure*}


\begin{figure*}[!tbh]
\begin{verbatim}
 1 int dotp(int *restrict a, int *restrict b, int n)
 2 {
 3   int sum = 0;
 4 
 5   #pragma omp target map(sum) map(to:a[:n],b[:n])
 6   #pragma omp teams distribute reduction(+:sum)
 7   for (int i=0; i<n; i++)
 8     sum += a[i] * b[i];
 9 
10   return sum;
11 }
\end{verbatim}
\caption{ \textbf {Use a separate target construct to map reduction variables} -- \small
          The variable \texttt{sum} is private in the teams region, but now mapped
          in the target region.
        }
\label{figure:chapter6-targetteams-reduction-v2}
\end{figure*}


\begin{figure*}[!b]
\begin{verbatim}
 1 #include <stdlib.h>
 2 void func(float a[1024], float b[1024], int t[1024])
 3 {
 4   #pragma omp target map(a, b, t) // Map-enter
 5   {
 6     int i;
 7 
 8     for (i=0; i<1024; i++)
 9       t[i] = rand()%1024;
10 
11     for (i=0; i<1024; i++)
12       a[i] =  b[t[i]];
13 
14   } // End of target, map-exit
15 }
\end{verbatim}
\caption{ \textbf {Example of the map clause} -- \small
          Copies occur for the arrays \texttt{a}, \texttt{b}, and 
          \texttt{t} at the entry to and exit from the target region.
         }
\label{figure:chapter6-map-v1}
\end{figure*}


\begin{figure*}[!tb]
\begin{verbatim}
 1 #include <stdlib.h>
 2 void func(float a[1024], float b[1024], int t[1024])
 3 {
 4   #pragma omp target map(from:a) map(to:b) \
 5                      map(alloc:t) // Map-enter
 6   {
 7     int i;
 8 
 9     for (i=0; i<1024; i++)
10       t[i] = rand()%1024;
11 
12     for (i=0; i<1024; i++)
13       a[i] =  b[t[i]];
14 
15   } // End of target, map-exit
16 }
\end{verbatim}
\caption{ \textbf {Example of the map clause with map-types} -- \small
          Eliminate superfluous copies by using map-types.
         }
\label{figure:chapter6-map-v2}
\end{figure*}


\begin{figure*}[!tb]
\begin{verbatim}
 1 #include <stdlib.h>
 2 #include <stdio.h>
 3 void func(float a[1024], float b[1024], int t[1024])
 4 {
 5   #pragma omp target data map(from:a) map(to:b) \
 6                           map(alloc:t) // Map-enter
 7   {
 8     #pragma omp target map(always,from:t) // Map-enter
 9     for (int i=0; i<1024; i++) {
10       t[i] = rand()%1024;
11     } // Map-exit
12 
13     for (int i=0; i<1024; i++)
14        printf("t[%d]=%d\n", i, t[i]);
15 
16     #pragma omp target map(a,b,t) // Map-enter
17     for (int i=0; i<1024; i++) {
18       a[i] =  b[t[i]];
19     } // Map-exit
20 
21   } // End of target data, map-exit
22 }
\end{verbatim}
\caption{ \textbf {Example of a variable appearing in nested map clauses} -- \small
          There is only one instance of a variable in an accelerator's
          address space.
         }
\label{figure:chapter6-map-v3}
\end{figure*}


\begin{figure*}[!tb]
\begin{verbatim}
 1 typedef struct { int x, y, size, *p; } Stype;
 2 extern Stype S;
 3 extern int f1(int,int), f2(int*,int), f3(Stype *);
 4 #pragma omp declare target to(f1, f2, f3) 
 5 
 6 void foo1(Stype S)
 7 {
 8    #pragma omp target
 9    f1(S.x, S.y);
10 
11    #pragma omp target map(S.x, S.y)
12    f1(S.x, S.y);
13 
14    #pragma omp target map(S, S.p[:S.size])
15    f2(S.p, S.size);
16 
17    #pragma omp target map(S.x)
18    f3(&S);
19 
20    #pragma omp target data map(S.x)
21    {
22       #pragma omp target map(S.y) // Error
23       f1(S.x, S.y);
24    }
25 }
\end{verbatim}
\caption{ \textbf {Example of mapping structure members} -- \small
          Structure members may appear in map clauses and array sections with some
          restrictions.
         }
\label{figure:chapter6-mapstruct}
\end{figure*}


\begin{figure*}[!tb]
\begin{verbatim}
 1 void saxpy(float *restrict y, float *restrict x, float a, int n)
 2 {
 3   #pragma omp target teams map(y[:n]) map(to:x[:n]) 
 4   {
 6     #pragma omp distribute 
 7     for (int i=0; i<n; i++) {
 8         y[i] =  y[i] + a*x[i];
 9     } // End of distribute
10   } // End of target teams
11 }
\end{verbatim}
\caption{ \textbf {Example of default data-mapping attribute rules} -- \small
          The pointer variables \texttt{x} and \texttt{y} are \texttt{private}. The 
          scalar variables \texttt{a} and \texttt{n} are \texttt{firstprivate}.
         }
\label{figure:chapter6-defaultmap-v1}
\end{figure*}


\begin{figure*}[!tb]
\begin{verbatim}
 1 float dotp(float *restrict y, float *restrict x, int n)
 2 {
 3   float sum = 0.0;
 4   #pragma omp target map(y[:n]) map(to:x[:n])
 5   {
 6     for (int i=0; i<n; i++)
 7       sum += y[i] * x[i];
 8   } // End of target
 9
10   return sum;
11 }
\end{verbatim}
\caption{ \textbf {Example of problems with implicit firstprivate variables} -- \small
          Because the variable \texttt{sum} is \texttt{firstprivate}, the computed value
          of \texttt{sum} is lost at the end of the target region.
         }
\label{figure:chapter6-defaultmap-v2}
\end{figure*}


\begin{figure*}[!tb]
\begin{verbatim}
 1 #include <stdlib.h>
 2 void f()
 3 {
 4    char *p, *q, A[128];
 5    extern void f1(char *);
 6    #pragma omp declare target to(f1) 
 7 
 8    p = malloc(1024);
 9    q = A;
10 
11    #pragma omp target data map(p[:1024], A)
12    {
13      #pragma omp target map(p[:0])
14      f1(p);
15 
16      #pragma omp target // Implicit map(q[:0])
17      f1(q);
18 
19      #pragma omp target map(p)
20      f1(p); // Error 
21    }
22    free(p);
23 }
\end{verbatim}
\caption{ \textbf {Example of C/C++ pointers as zero-length array sections} -- \small
          Pointer variables are implicitly treated as pointer-based zero-length array
          sections in target regions.
         }
\label{figure:chapter6-zerolength}
\end{figure*}


\begin{figure*}[!tb]
\begin{verbatim}
 1 int Lastpos = 0;
 2 extern char Buf[128];
 3 #pragma omp declare target(Lastpos, Buf)
 4 
 5 extern int F(int, int);
 6 #pragma omp declare target to(F)
 7 
 8 #pragma omp declare target
 9 int State = -1;
10 extern int search(char *);
11 
12 void find_state(char c)
13 {
14   int pos;
15   Buf[Lastpos] = c;
16   pos = search(Buf);
17   Lastpos = pos;
18   State = F(Lastpos, pos);
19 }
20 #pragma omp end declare target
21 
22 void process_input(char c)
23 {
24   #pragma omp target firstprivate(c)
25   find_state(c);
26 }
\end{verbatim}
\caption{ \textbf {Example of the declare target directive} -- \small
          Various forms of the directive all have the same effect.
         }
\label{figure:chapter6-decltarg}
\end{figure*}


\begin{figure*}[!tb]
\begin{verbatim}
 1 float Vector[1024];
 2 #pragma omp declare target link(Vector)
 3 
 4 #pragma omp declare target
 5 extern float F(float , float);
 6 int compute(float a)
 7 {
 8    for (int i=0; i<1024; i++)
 9       Vector[i] = F(Vector[i], a);
10 }
11 #pragma omp end declare target
12 
13 int update_vector(float a)
14 {
15   #pragma omp target map(Vector) firstprivate(a)
16   compute(a);
17 }
\end{verbatim}
\caption{ \textbf {Example of the link clause on a declare target directive} -- \small
          Variables appearing in the \texttt{link} clause are \emph{globally linked}.  
          They must be mapped before they are referenced in a mapped function.
         }
\label{figure:chapter6-link}
\end{figure*}


\begin{figure*}[!tbhp]
\begin{verbatim}
 1 #define N (1024*1024)
 2 double A[N], B[N];
 3 extern double F(double * restrict);
 4 
 5 void G(double c, double d)
 6 {
 7   double e;
 8   #pragma omp target data map(B)
 9   {
10     #pragma omp target map(B) map(always,from:A) \
11                        firstprivate(c,d)
12     for (int i=0; i<N; i++)
13       A[i] =  B[i] * c + d;
14 
15     e = F(A);
16 
17     #pragma omp target map(B) firstprivate(e)
18     for (int i=0; i<N; i++)
19       B[i] =  B[i] / e;
20 
21   } // End of target data
22 }
\end{verbatim}
\caption{ \textbf {Example of a target data construct} -- \small
          The array variable \texttt{B} is mapped once to an accelerator
          across two target regions.
         }
\label{figure:chapter6-target-data-v1}
\end{figure*}


\begin{figure*}[!tb]
\begin{verbatim}
 1 #define N (1024*1024)
 2 extern void update_boundary(double *, int, int);
 3 double B[N];
 4 #pragma omp declare target(B)
 5 
 6 void G(double *restrict B, double e, int n, int l)
 7 {
 8   #pragma omp target update from(B[0:l],B[n-1-l:l])
 9   update_boundary(B, n, l);
10   #pragma omp target update to(B[0:1],B[n-1-l:l])
11 
12   #pragma omp target
13   for (int i=0; i<n; i++)
14     B[i] =  B[i] / e;
15 }
\end{verbatim}
\caption{ \textbf {Example of the target update construct} -- \small
          The array variable \texttt{B} is globally mapped.  The target update
          construct is used make elements at the start and the end the
          array \texttt{B} consistent between the host and the accelerator.
         }
\label{figure:chapter6-target-update-v1}
\end{figure*}


%\begin{figure*}[!tbh]
%\begin{verbatim}
% 1 #pragma omp target data map(b)
% 2 {
% 3   #pragma omp target map(to:c,d,a)
% 4   for (i=0; i<count; count++)
% 5   {
% 6     a[i] =  b[i] * c + d;
% 7   }
% 8
% 9   e = F(a);
%10   for (i=0; i<count; i+=step)
%11      b[i] = G(b[i], step);
%12
%13   #pragma omp target update to(b)
%11
%12   #pragma omp target map(to:e)
%13   for (i=0; i<count; count++)
%14   {
%15     b[i] =  b[i] + e;
%16   }
%17 } // End of target data
%}
%\end{verbatim}
%\caption{ \textbf {Code fragment with a target update directive} -- \small
%          The variable b is mapped to a device data environment once
%          across two target regions and updated with the host value
%          between the target regions.
%         }
%\label{figure:chapter6-target-update}
%\end{figure*}


\begin{figure*}[!tb]
\begin{verbatim}
 1 class myArray {
 2   int length;
 3   double *ptr;
 4 
 5   void allocate(int l) {
 6     double *p = new double[l];
 7     ptr = p;
 8     length = l;
 9     #pragma omp target enter data map(alloc:p[0:length])
10   }
11 
12   void release() {
13     double *p = ptr;
14     #pragma omp target exit data map(release:p[0:length])
15     delete[] p;
16     ptr = 0;
17     length = 0;
18   }
19 };
\end{verbatim}
\caption{ \textbf {C++ Example of the target enter and exit data constructs} -- \small
          The \texttt{allocate()} method will execute a map-enter phase for the
          dynamically allocated memory pointed to by \texttt{p}.  The \texttt{release()}
          method will execute the corresponding map-exit phase.
         }
\label{figure:chapter6-target-enter-exit-data}
\end{figure*}


\begin{figure*}[!tb]
\begin{verbatim}
1 void G(char S[128], int v)
2 {
3   extern int mutate(char *s, int);
4   while (!v) {
5     #pragma omp target enter data map(to:S)
6     v = mutate(S, v);
7   }
8   #pragma omp target exit data map(delete:S)
9 }
\end{verbatim}
\caption{ \textbf {Example of the delete map-type} -- \small
          Regardless of its reference count, remove \texttt{S} from an     
          accelerator's device data environment.
         }
\label{figure:chapter6-delete-maptype}
\end{figure*}


\begin{figure*}[!tb]
\begin{verbatim}
 1 extern int max(int,int);
 2 #pragma omp declare target(max)
 3 void F(char *v, short *restrict s, int n)
 4 {
 5   int i;
 6 
 7   #pragma omp target nowait map(v[0:n])
 8   for (i=0; i<n; i++)
 9     v[i] = max(v[i],0);
10 
11   for (i=0; i<n; i++)
12     s[i] = max(s[i],0);
13   #pragma omp taskwait
14 
15   for (i=0; i<n; i++)
16     s[i] = s[i] - v[i];
17 }
\end{verbatim}
\caption{ \textbf {Example using the nowait clause } -- \small
          Execute the target region on an accelerator in 
          parallel with the code executing on the host.
         }
\label{figure:chapter6-nowait-v2}
\end{figure*}


\begin{figure*}[!tb]
\begin{verbatim}
 1 extern void h0(int*, int);
 2 extern void h1(int*, int);
 3 extern void t1(int*, int*, int);
 4 #pragma omp declare target(t1)
 5 void F(int *a, int *b, int n)
 6 {
 7   #pragma omp parallel num_threads(2)
 8   #pragma omp single
 9   {
10   #pragma omp target enter data map(to:a[:n]) \
11               nowait depend(out:a[:n]) // t0
12 
13   #pragma omp task depend(out:b[:n])
14   h0(b, n);
15 
16   #pragma omp target map(to:b[:n]) \
17               nowait depend(in:b[:n]) depend(inout:a[:n])
18   t1(a, b, n);
19 
20   #pragma omp task depend(in:b[:n])
21   h1(b,n);
22 
23   #pragma omp target exit data map(from:a[:n]) \
24               depend(in:a[:n]) // t2
25   }
26 }
\end{verbatim}
\caption{ \textbf {Example using the nowait and depend clauses } -- \small
          Use the \texttt{depend} and \texttt{nowait} clauses to execute target
          tasks in parallel with other host tasks.
         }
\label{figure:chapter6-nowait-depend}
\end{figure*}


\begin{figure*}[!tb]
\begin{verbatim}
 1 #include <omp.h>
 2 extern int a[1024]; int Work(int *, int, int);
 3 #pragma omp declare target to(a, Work)
 4 
 5 void F()
 6 {
 7   int defdev = omp_get_default_device();
 8   int numdev = omp_get_num_devices();
 9 
10   for (int i=0; i<numdev; i++) {
11     omp_set_default_device(i);
12     #pragma omp target update to(a) nowait
13   }
14   omp_set_default_device(defdev);
15   #pragma omp taskwait
16 
17   for (int i=0; i<numdev; i++) {
18     #pragma omp target device(i) nowait
19     Work(a,i,numdev);
20   }
21 
22   if (numdev == 0) Work(a,0,1);
23   #pragma omp taskwait
24 }
\end{verbatim}
\caption{ \textbf {Example of the device clause and related runtime functions} -- \small
          The variable \texttt{a} is updated with the host's value on all devices
          and then the function \texttt{Work()} is executed by all devices.
         }
\label{figure:chapter6-device}
\end{figure*}


\begin{figure*}[!tb]
\begin{verbatim}
1 #define MB (1024*1024)
2 extern void Work(float*, int);
3 #pragma omp declare target(Work)
4 void F(float * restrict a, int n)
5 {
6    #pragma omp target if(n > MB) map(a[:n])
7    Work(a,n);
8 }
\end{verbatim}
\caption{ \textbf {Example of an if clause on the target construct} -- \small
          If \texttt{n} is greater than a threshold, execute the target region on the
          default accelerator.  Otherwise, execute the region on the host device.
         }
\label{figure:chapter6-if}
\end{figure*}


\begin{figure*}[!tb]
\begin{verbatim}
 1 #include <omp.h>
 2 void *init(int n, int dev)
 3 {
 4   char *dptr = omp_target_alloc(dev, n);
 5 
 6   #pragma omp target is_device_ptr(dptr) device(dev) 
 7   for (int i=0; i<n; i++)
 8     dptr[i] = i;
 9 
10   return (void*)dptr;
11 }
\end{verbatim}
\caption{ \textbf {Example of the is\_device\_ptr clause} -- \small
          The device pointer variable \texttt{dptr} must appear in
          the \texttt{is\_device\_ptr} clause to 
          de-reference it in the target region.
        }
\label{figure:chapter6-isptr}
\end{figure*}


\begin{figure*}[!tb]
\begin{verbatim}
 1 int A[1024];
 2 #pragma omp declare target to(A)
 3 extern int AccelFunc(void *);
 4 
 5 int Func()
 6 {
 7   int err;
 8   #pragma omp target data map(err) use_device_ptr(A)
 9   err = AccelFunc(A); // Requires device address of A
10   return err;
11 }
\end{verbatim}
\caption{ \textbf {Example of the use\_device\_ptr clause} -- \small
          Replace the reference to the host address of \texttt{A} in the lexical
          scope of the \texttt{target data} construct with
          the device address of \texttt{A}.
        }
\label{figure:chapter6-useptr}
\end{figure*}


\begin{figure*}[!tb]
\begin{verbatim}
 1 #include <omp.h>
 2 #include <stdlib.h>
 3 typedef struct item {struct item *next; int v; } item_t;
 4 void *copy_list2dev(item_t *list)
 5 {
 6   int i, count=0;
 7   int dev  = omp_get_default_device();
 8   int host = omp_get_initial_device();
 9   item_t *src = NULL, *dst = NULL;
10 
11   if (list == NULL) return NULL;
12   for (src=list; src; src=src->next)
13     count++;
14   dst = omp_target_alloc(count*sizeof(item_t), dev);
15   
16   for (src=list, i=0; src; src=src->next, i++)
17     omp_target_memcpy(dst, src, sizeof(item_t),
18                       i*sizeof(item_t), 0,
19                       dev, host);
20                       
21   #pragma omp target is_device_ptr(dst)
22   {
23     for (i=0; i<count-1; i++)
24       dst[i].next = &dst[i+1];
25     dst[i].next = NULL;
26   } 
27   return (void*)dst;
28 } 
\end{verbatim}
\caption{ \textbf {Copy a linked list to device memory} -- \small
          Copy a linked list from the host to dynamically allocated
          device memory.
         }
\label{figure:chapter6-alloc}
\end{figure*}


\begin{figure*}[!tb]
\begin{verbatim}
 1 #include <omp.h>
 2 void stream(float *restrict a, int n, int chunk, int dev)
 3 {
 4   int size = sizeof(float)*chunk;
 5   float *devptr = omp_target_alloc(size, dev);
 6 
 7   for (int i=0; i<n; i+=chunk)
 8   {
 9      omp_target_associate_ptr(&a[i], devptr, size, 0, dev);
10 
11      #pragma omp target map(always,tofrom:a[i:chunk]) device(dev)
12      for (int j=i; j<i+chunk; j++)
13         a[j] = 1/(1+a[j]);
14 
15      omp_target_disassociate_ptr(&a[i], dev);
16   }
17   omp_target_free(devptr, dev);
18 }
\end{verbatim}
\caption{ \textbf {Map host memory to dynamically allocated device memory} -- \small
          Iteratively associate a smaller device memory buffer with a section
          of a larger \texttt{a} buffer.
         }
\label{figure:chapter6-assoc}
\end{figure*}


\begin{figure*}[!tb]
\begin{verbatim}
 1 #include <stdio.h>
 2 #include <omp.h>
 3 int copy_2d(void *dst, void *src, int dst_dev, int src_dev,
 4             int sz, int vol_sz, int offset)
 5 {
 6    const int num_dims = 2;
 7    const int vol_dims[2] = {vol_sz, vol_sz};
 8    const int dst_dims[2] = {sz, sz};
 9    const int src_dims[2] = {sz, sz};
10    const int dst_offset[2] = {offset, offset};
11    const int src_offset[2] = {0, 0};
12 
13    return omp_target_memcpy_rect(dst, src, sizeof(char),
14          num_dims,
15          vol_dims,
16          dst_offset, src_offset,
17          dst_dims, src_dims,
18          dst_dev, src_dev);
19 }
\end{verbatim}
\caption{ \textbf {Copy a sub-matrix from a source matrix to a destination matrix - part 1} -- \small
          Two-dimensional square matrices are assumed.
          Copy a sub-matrix from \texttt{src[0][0]} to \texttt{dst[offset][offset]}.
         }
\label{figure:chapter6-copy-volume}
\end{figure*}


\begin{figure*}[!tb]
\begin{verbatim}
21 #define N 8
22 void main()
23 {
24   int dst_dev = omp_get_initial_device();
25   int src_dev = omp_get_default_device();
26   unsigned char DST[N][N];
27   unsigned char (*SRC)[N] = omp_target_alloc(N*N, src_dev);
28 
29   #pragma omp target is_device_ptr(SRC) nowait
30   for (int i=0; i<N; i++)
31     for (int j=0; j<N; j++) SRC[i][j] = 1;
32 
33   for (int i=0; i<N; i++)
34     for (int j=0; j<N; j++) DST[i][j] = 0;
35 
36   #pragma taskwait
37   copy_2d(DST, SRC, dst_dev, src_dev, N, 4, 2);
38 
39   omp_target_free(SRC, src_dev);
40 
41   for (int i=0; i<N; i++) {
42     for (int j=0; j<N; j++) printf("%d" , DST[i][j]);
43     printf("\n");
44   }
45 }
\end{verbatim}
\caption{ \textbf {Copy a sub-matrix from a source matrix to a destination matrix - part 2} -- \small
          Allocate and initialize an $8x8$ \texttt{SRC} matrix on an accelerator and fill it with $1$.
          Initialize an $8x8$ \texttt{DST} matrix on the host and fill it with $0$.
          Copy a $4x4$ sub-matrix from \texttt{SRC[0][0]} to \texttt{DST[2][2]}.
         }
\label{figure:chapter6-copy-volume-p2}
\end{figure*}


\begin{figure*}[!tb]
\begin{verbatim}
00000000
00000000
00111100
00111100
00111100
00111100
00000000
00000000
\end{verbatim}
\caption{ \textbf {Example output from the sub-matrix copy program } -- \small
          This is the output from the program in Figure~\ref{figure:chapter6-copy-volume}
          and Figure~\ref{figure:chapter6-copy-volume-p2}.
          A $4x4$ sub-matrix from the \texttt{SRC} matrix was copied into the center of the \texttt{DST} matrix.
         }
\label{figure:chapter6-copy-volume-output}
\end{figure*}
