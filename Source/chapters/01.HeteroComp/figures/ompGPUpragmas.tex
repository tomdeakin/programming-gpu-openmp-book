%%%%%%%%%%%%%%%%%%%%%%%%%%%%%%%%%%%%%%
%%  
%% The table contains OpenMP pragmas we'll cover in the book
%%
%% To use this in one of our chapters, use the latex command
%%         \begin{table}[h!]
%%          \centering
%%          \caption{Your caption}
%%           \label{YourLabel}
%%           \input{\SubDir/CommonCoreTab.tex}
%%           \end{table}
%%
%%  where the macro \SubDir is defined at the top of each chapter and 
%%  specifies where the source for the book resides.   
%%
%%%%%%%%%%%%%%%%%%%%%%%%%%%%%%%%%%%%%%

\begin{tabular}{|l|l|l|l|}
\hline
\textbf{OpenMP pragma,}  & \textbf{Concepts} & Chapter & Structure \\
\hline
%============================================================
\Code{#pragma omp target}                & offload onto a device  & \S\ref{chapter:target} & Core \\ 
\hline 
\Code{#pragma omp teams}                & Create a league of teams & \S\ref{sec:teams} & Core \\
\hline
\Code{#pragma omp distribute}            & Map loops onto a league & \S\ref{ssec:teams_distribute} & Advanced (on its own, favor combined) \\
\hline
\Code{#pragma omp target data}            & Create a device data region & \S\ref{ssec:target_data} & Core \\
\hline
\Code{#pragma omp target enter data}  & Enter a device data region & \S\ref{ssec:target_enter_exit_data} & Core \\
\hline
\Code{#pragma omp target exit data}  & Exit a device data region & \S\ref{ssec:target_enter_exit_data} & Core \\
\hline
\Code{#pragma omp target update}  & Update data between host/device & \S\ref{ssec:target_update} & Core \\
\hline
\Code{#pragma omp loop}  & Concurrent loop iterations & \S\ref{sec:loop} & Core \\
\hline
\Code{#pragma omp requires} & OMP features required & \S\ref{sec:usm} & Core/Advanced \\
\hline
\Code{#pragma omp declare target} & declares items are mapped to a device & \S\ref{ssec:declare_target} & Advanced \\
\hline
\Code{#pragma declare mapper} & user defined mapers & \S\ref{sec:mapper} & Advanced \\
\hline
\Code{#pragma omp interop} & Interoperability with foreign contexts & XX & Advanced \\
\hline
\Code{teams loop}    & combined construct & \S\ref{sec:loop} & Core \\
\hline
\Code{teams distribute}  & combined  construct & \S\ref{sec:teams_distribute} & Core \\
\hline
\Code{teams distribute simd}  & combined  construct & XX & Advanced (not useful?) \\
\hline
\Code{teams distribute parallel for} & combined  construct & XX & Advanced \\
\hline
\Code{teams distribute parallel for simd} & combined  construct & \S\ref{sec:bud} & Core \\
\hline
\Code{target parallel}    & combined construct & XX & Advanced \\
\hline
\Code{target simd}    & combined construct & XX & Advanced \\
\hline
\Code{target parallel for}    & combined construct & XX & Advanced \\
\hline
\Code{target parallel for simd}    & combined construct & XX & Advanced \\
\hline
\Code{target parallel loop}    & combined construct & XX & Advanced (N/A?) \\
\hline
\Code{target teams }    & combined construct & XX & Advanced \\
\hline
\Code{target teams loop}    & combined  construct & \S\ref{sec:loop} & Core \\
\hline
\Code{target teams distribute}  & combined  construct & XX & Advanced \\
\hline
\Code{target teams distribute simd}  & combined  construct & XX & Advanced (N/A?) \\
\hline
\Code{target teams distribute parallel for} & combined  construct & XX & Advanced \\
\hline
\Code{target teams distribute parallel for simd} & combined  construct & \S\ref{sec:bud} & Core \\
\hline
\end{tabular}

Comment: best practice often favors the combined constructs (distribute must be strictly nested).

Comment: often the combined construct falls out naturally from explaining what the bits mean.


