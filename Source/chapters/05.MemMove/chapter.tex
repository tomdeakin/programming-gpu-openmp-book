%%%%%%%%%%%%%%%%%%%%%%%%%%%%%%%%%%%%%%
%%  
%% Chapter 5: Memory Movement
%%
%%%%%%%%%%%%%%%%%%%%%%%%%%%%%%%%%%%%%%

\def\ArtDir{05.MemMove/figures}%

\chapter{Memory movement}
\label{chapter:memory}

\section{Recap implicit data movement}
\begin{itemize}
  \item So far only could work with scalars and stack arrays.
  \item Now we explain heap arrays.
  \item Recall firstprivate clause.
  \item The pointer itself will be mapped firstprivate.
  \item But the data needs to be mapped\dots
\end{itemize}

\section{Explicit data sharing with the map clause}
\subsection{Array syntax: important as non-intuitive in C}
\begin{itemize}
  \item Introduce OpenMP array notation.
  \item array[start:length].
  \item It is not array[start:end] in C.
  \item Follows pointer size rules, so don't have to worry about sizeof(T).
  \item Array notation in Fortran.
\end{itemize}

\subsection{The map clause}
\label{ssec:map_clause}
\begin{itemize}
  \item Map clause on a target directive.
  \item List notation of things to map.
  \item to, from and tofrom.
  \item Recall the scoped behaviour: from transfer happens on the closing curly brace.
  \item alloc and delete and release.
  \item Pointer attachment from OpenMP 5.0
  \item Some mention of reference counting. Things only transfer based on that. Read the spec for the details.
\end{itemize}

\subsection{Example: Vector add on the heap}
\begin{itemize}
  \item Similar to vector add example so far.
  \item Arrays now allocated on the heap.
  \item Show use of map(tofrom) clauses to copy the data to the device, run in parallel, then copy back at end.
\end{itemize}

\section{Reductions and mapping back result.}
\label{sec:target_reductions}
\begin{itemize}
  \item In OpenMP 5, the reduction result is brought back automatically.
  \item Need to add a note that in OpenMP 4.5, need to use the map clause to do a reduction on the device.
  \item map(tofrom: scalar)
  \item Explain that without this it is firstprivate and not copied back.
  \item Explain interaction of reduction clause with teams and threads and simd in the OpenMP hierarchy of parallelism.
\end{itemize}

\subsection{Example: Pi}
\begin{itemize}
  \item Show pi reduction example.
  \item Has no data arrays to map.
  \item But need to map the reduction result.
\end{itemize}

\section{Data regions}
\begin{itemize}
  \item Motivation: so far expressed data movement for each target region.
  \item Wasteful as can often just leave it there between target regions.
  \item Multiple target regions, leaving data resident.
  \item Motivating example: multiple subsequent transformations to data, and/or iterative solve.
  \item Wasteful as only want to do the mapping: both writing and in terms of optimal data movement.
  \item Idea of target data region. Inherited by target regions. Data already exists on the device.
\end{itemize}

\subsection{Target data directive}
\label{ssec:target_data}
\begin{itemize}
  \item Scoped target data region.
  \item Useful for lots of kernels running after each other.
  \item Show an example of this.
  \item No need to map on the target regions themselves.
  \item Downsides are the structured block requirements.
  \item From destination determined at start of region.
\end{itemize}

\subsection{Target enter/exit data}
\label{ssec:target_enter_exit_data}
\begin{itemize}
  \item Change target data environment in unstructured way.
  \item Expect will need to talk about reference counting again.
  \item Useful because don't need scoped. But behaves similarly.
  \item Example use just after every allocation.
  \item Means this helps with C++ class data in constructors.
  \item Example: Show the same target data example with the target enter/exit data.
\end{itemize}

\subsection{Target update directive}
\label{ssec:target_update}
\begin{itemize}
  \item Need to get data back in the middle of target data region.
  \item target update directive.
  \item Uses same map clauses as before.
  \item Example: convergence, or halo exchange.
\end{itemize}

\subsection{Pointer swapping}
\begin{itemize}
  \item Pointers are mapped firstprivate and zero-length arrays.
  \item Updates reference count.
  \item Doesn't need to copy any data, and don't need to specify length as already mapped.
  \item Will be remapped/reevaluated on each target region.
  \item How do you do pointer swapping in a target data region.
  \item How do you do pointer swapping when you use target enter/exit.
\end{itemize}

\section{Example: Jacobi or CG or simple linear solver.}
\begin{itemize}
  \item Biggest example, showing off everything so far.
  \item Key requirements: iterative and pointer swap to motivate cutting down memory movement
  \item Use target enter/exit data regions.
  \item Use pointer swapping.
  \item Use target update.
  \item Uses parallelism, with/without reduction clause.
\end{itemize}

\section{Unified shared memory}
\label{sec:usm}
\begin{itemize}
  \item OpenMP 5 add requires directive
  \item Which unified shared memory is.
  \item Data moves back and from at will (page fault).
  \item Requires hardware support.
  \item Pros: each to get started, as don't have to map.
  \item Cons: parallelism is the easy bit, it's the data movement which is important.
  \item In 5.0, limited performance tuning opportunities.
  \item Mapping gives you the control for optimal performance.
\end{itemize}


\section{Advanced parts}
\subsection{Declare target}
\label{ssec:declare_target}
\begin{itemize}
  \item map global variables
\end{itemize}


\subsection{User-defined mappers}
\label{sec:mapper}
\begin{itemize}
  \item From OpenMP 5.0
  \item Allow user to define how structures are mapped.
  \item Useful for complex structures
  \item Example: struct with a pointer variable. Declare mapper of struct which also maps the pointer.
\end{itemize}

\subsection{Allocator and memcpy APIs and related clauses}
\label{sec:alloc_apis}
\begin{itemize}
  \item \Code{omp\_target\_alloc()}
  \item \Code{omp\_target\_free()}
  \item \Code{use\_device\_ptr}
  \item allocate, test, and move memory using the APIs instead of the pragmas.
\end{itemize}



