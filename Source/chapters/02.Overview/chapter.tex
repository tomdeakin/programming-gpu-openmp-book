%%%%%%%%%%%%%%%%%%%%%%%%%%%%%%%%%%%%%%
%%  
%% Chapter 2: OpenMP overview
%%
%%%%%%%%%%%%%%%%%%%%%%%%%%%%%%%%%%%%%%

\def\ArtDir{02.Overview/figures}%



\chapter[OpenMP overview]{OpenMP overview}
\label{chapter:OMPoverview}

OpenMP is old.  The first release (OpenMP 1.0 for Fortran) was in 1997.  
 It started as the simplest way we could think of for applications programmers 
 in high performance computing to write multithreaded code.  The focus
was on parallelizing loops in Fortran and the full OpenMP definition took only 40 pages.

At the time we are writing this book, OpenMP is at release 5.1.  The specification covers
Fortran, C and C++.  It includes multiple ways to parallelize loops, explicit tasks, 
SIMD parallelism (to vectorize loops), manage complex memory hierarchies, and much 
more (including programming GPUs, of course).

With OpenMP 1.0, the entire language could be covered in a single day.  Now, it takes months if
not years to master the entire specification.  New versions of the specification come out roughly every 
two years.  Hence, once the language is mastered, something new comes along to expand its scope 
even future.

The situation with OpenMP would be unexpectedly frustrating, if not for one simple fact.  Nobody
uses the entire language.  OpenMP programmers learn the foundational terminology of OpenMP and 
a simplified common core of the language elements of OpenMP.  Most of the time, that common core
is all a programmer needs.  When additional features of OpenMP are needed, programmers look them up in 
the specification.  By learning and then using just those parts of OpenMP needed to solve a programming 
problem, the complexity of OpenMP is manageable and that inner-simplicity that motivated us at the beginning
shines through.

In this chapter, we will cover the essential elements of OpenMP that all programmers use.   We will be brief
and expose them through an investigation of the three fundamental design patterns of multithreaded programming
with OpenMP; loop-level parallelism, SPMD, and Task-parallelism.   Then we will go back through the core
elements of OpenMP and explain how they are organized around the behavior of both implied and explicit tasks.
This is essential since we can't explain GPU programming in OpenMP without using those details.  We 
then close the chapter with an overview of our journey into GPU programming in OpenMP.  

\section{The fundamental design patterns of OpenMP}

I will describe the fork-join model and the concept of thread teams.  I'll probably talk about 
numbers of threads, environment variables to control threads and a few other concepts.

Then I'll list the three patterns we'll cover and the program we'll use to explore them.  
Take a look at figure~\ref{code:PiSeq}.  It's really pretty.  This is the serial or sequential code for the pi
program. 




\begin{CodeExample}%
{\textbf{Sequential $\pi$ program} --\small This program carries out a numerical integration 
of a definite integral selected such that the result should be the number $\pi$.
}%
{code:PiSeq}
\begin{lstlisting}

#include <stdio.h>
#include <omp.h>
static long num_steps = 1024*1024*1024;

int main()
{
   double x, pi, step, sum = 0.0;
   step = 1.0 / (double) num_steps;

   for (int i = 0; i < num_steps; i++) {
      x = (i + 0.5) * step;
      sum += 4.0 / (1.0 + x * x);
   }

   pi = step * sum;
   printf("pi = %lf, with %ld steps\n ", pi, num_steps);
}
\end{lstlisting}
\end{CodeExample}



\subsection{The SPMD pattern}

 have an SPMD parallel version of this code in figure~\ref{code:piSPMD}.
 
 
 
\begin{CodeExample}%
{\textbf{OpenMP $\pi$ program, SPMD pattern} --\small This program carries out a numerical integration 
of a definite integral selected such that the result should be the number $\pi$.  This code uses the 
SPMD pattern.
}%
{code:PiSPMD}
\begin{lstlisting}
#include <stdio.h>
#include <omp.h>

static long num_steps = 1024*1024*1024;
int main ()
{
   int numthreads;
   double pi, step, full_sum = 0.0;
   step = 1.0 / (double) num_steps;

   #pragma omp parallel 
   {
      int id = omp_get_thread_num();
      double x, partial_sum = 0;

      #pragma omp single
         numthreads = omp_get_num_threads();

      for (int i = id; i < num_steps; i += numthreads) {
         x = (i + 0.5) * step;
         partial_sum += 4.0 / (1.0 + x*x);
      }
      #pragma omp critical
         full_sum += partial_sum;
    }
      
   pi = step * full_sum;
   printf("\n pi is %f with %d with threads \n ", pi, numthreads);
}	  

\end{lstlisting}
\end{CodeExample}



\subsection{The Loop level pattern}

\begin{CodeExample}%
{\textbf{Parallel $\pi$ program, Loop parallelism pattern} --\small This program carries out a numerical integration 
of a definite integral selected such that the result should be the number $\pi$.
}%
{code:PiLoop}
\begin{lstlisting}

#include <stdio.h>
#include <omp.h>
static long num_steps = 100000000;

int main ()
{
   double x, pi, step, sum = 0.0;
   int numthreads;
   step = 1.0 / (double) num_steps;

    #pragma omp parallel  
    {
         #pragma omp single
              numthreads =omp_get_num_threads();

         #pragma omp for private(x) reduction(+:sum) 
            for (int i = 0; i < num_steps; i++){
                 x = (i + 0.5) * step;
                 sum = sum + 4.0 / (1.0 + x*x);
	     }
     }
    pi = step * sum;
    printf("\n pi is %f with %d threads\n", pi,numthreads);
}	  
\end{lstlisting}
\end{CodeExample}



\subsection{Divide and conquer}

While we are at it, we have figure~\ref{code:PiLoop} figure~\ref{code:PiSeqDivCon} figure~\ref{code:PiParDivCon}


\begin{CodeExample}%
{\textbf{Sequential $\pi$ program, divide and conquer} --\small This program carries out a numerical integration 
of a definite integral selected such that the result should be the number $\pi$.
}%
{code:PiSeqDivCon}
\begin{lstlisting}

#include <omp.h>
#include <stdio.h>
static long num_steps = 1024*1024*1024;
#define MIN_BLK 1024*256
double pi_comp(int Nstart, int Nfinish, double step)
{   
   int iblk;
   double x, sum = 0.0, sum1, sum2;
   if (Nfinish - Nstart < MIN_BLK){
      for (int i = Nstart; i < Nfinish; i++) {
         x = (i + 0.5) * step;
         sum += 4.0 / (1.0 + x * x);
      }
   }
   else {
      iblk = Nfinish - Nstart;
      sum1 = pi_comp(Nstart, Nfinish - iblk/2, step);
      sum2 = pi_comp(Nfinish - iblk/2, Nfinish, step);
      sum = sum1 + sum2;
   }
   return sum;
}

int main () 
{
   int i;
   double step, pi, sum;
   step = 1.0 / (double) num_steps;

   sum = pi_comp(0, num_steps, step);
   pi = step * sum;

   printf(" for %ld steps pi = %f \n", num_steps, pi);
} 
\end{lstlisting}
\end{CodeExample}

\begin{CodeExample}%
{\textbf{Parallel $\pi$ program, divide and conquer with tasks} --\small This program carries out a numerical integration 
of a definite integral selected such that the result should be the number $\pi$.
}%
{code:PiParDivCon}
\begin{lstlisting}

#include <omp.h>
#include <stdio.h>
static long num_steps = 1024*1024*1024;
#define MIN_BLK 1024*256

double pi_comp(int Nstart,int Nfinish,double step)
{ 
   int iblk;
   double x, sum = 0.0, sum1, sum2;
   if (Nfinish - Nstart < MIN_BLK) {
      for (int i = Nstart; i < Nfinish; i++){
         x = (i + 0.5) * step;
         sum = sum + 4.0 / (1.0 + x*x); 
      }
   }
   else {
      iblk = Nfinish - Nstart;
      #pragma omp task shared(sum1)
         sum1 = pi_comp(Nstart, Nfinish - iblk/2, step);
      #pragma omp task shared(sum2)
         sum2 = pi_comp(Nfinish - iblk/2, Nfinish, step);
      #pragma omp taskwait
         sum = sum1 + sum2;
   }
   return sum;
}

int main ()
{
   double step, pi, sum;
   step = 1.0 / (double) num_steps;

      #pragma omp parallel 
      {
         #pragma omp single
         {
            printf("num threads=%d", omp_get_num_threads());
            sum = pi_comp(0, num_steps, step);
         }
      }
      pi = step * sum;
      printf(" for %ld steps pi = %f \n", num_steps, pi);
}  
\end{lstlisting}
\end{CodeExample}










\section{Tasks and the nature of OpenMP's execution model}


\section{Our journey ahead}



\begin{CodeExample}%
{\textbf{Matrix Multiplication program} --\small This program will multiply two matrices $A$ and $B$
to produce a third $C$ which has been set to zero, running on the target device.
All arrays have been previously allocated in stack memory.
}%
{code:matmulTarget}
\begin{lstlisting}
void matmul(int Ndim, int Mdim, int Pdim,
            float A[Ndim][Pdim], float B[Pdim][Mdim], float C[Ndim][Mdim]) {

  #pragma omp target teams distribute parallel for simd collapse(2)
  for (int i = 0; i < Ndim; i++) {
    for (int j = 0; j < Mdim; j++) {
      for(int k = 0; k < Pdim; k++) {
	C[i][j] += A[i][k] * B[k][i];
      }
    }
  }
}
\end{lstlisting}
\end{CodeExample}



\section{Notes: to be deleted later}

Note: Use C for examples in the book. Provide Fortran supplement.

What are the concepts we need to build on as we construct the rest of the book?  Here is my attempt at that list.  
\begin{itemize}
\item Directives, clauses, structured blocks, environment variables, library routines
\item a team of threads
\item Memory: stack vs heap
\item a memory consistency model and synchronization
\item The openMP data environment: private, firstprivate, shared.  
\item Do we need threadprivate?  No, we do not.
\item tasks and their role in understanding the core of OpenMP.  Initial tasks and implicit tasks
\item explicit tasks and how tasks interact with the data environment.
\item tied tasks, untied tasks, and nowait
\item we need mergeable tasks since ``A target task is mergable and untied''.
\item worksharing loops including collapse, reduction, and schedule clauses
\item for the schedule clauses, only cover static. 
\item design patterns: SPMD, loop-level parallelism, divide-and-conquer
\end{itemize}
I need everyone working on the book to go over this list and tell me what is missing.

\section{The plan for the chapter}

The running example will be matrix multiply.  I'd use the PI program but I've used that so much in
other places that its getting a bit old. Time to try something else.   Since we already decided to
close the chapter with matrix multiply, why not use it throughout the chapter?  This will be easy for 
SPMD and loop-level parallelism, but a bit trickier for divide-and-conquer.   Environment variables.

I will present a patterns based discussion mixed with a bit of history.

SPMD matrix multiply to introduce threads, fork join, the parallel construct, structured blocks, runtime-library-routines and
synchronization.   I will cover default rules of data sharing

Then loop-level parallelism.  Cover worksharing, data environment clauses, schedule(static) and reductions.  
collapse

Memory consistency, flushes, barriers, nowait.

Tasks in openMP.  Divide and conquer algorithms.   implied tasks, explicit tasks, tied vs untied
tasks, initial task, and mergeable tasks.  We need mergeable tasks since ``A target task is mergable and untied''.


\section{Set the Background/Context for the book.}

Starts with the {\bf problem} of trying to program a GPU.
Overview will shows everything a reader might need to review before the start on this book.

Will teach OpenMP 5.0. There are few relevant changes from 4.5, and we will call out any differences.

\section{Quick crash course in OpenMP}
This section will give the reader a very concise introduction into enough OpenMP for us to build on to explain the target directives to program GPUs.
\begin{itemize}
  \item What OpenMP shared memory is: cores share memory.
  \item Annotate code with compiler directives.
  \item API calls.
\end{itemize}

\subsection{Fork-join parallelism}
\label{ssec:fork_join}
\begin{itemize}
  \item Parallel: threads in a team.
  \item Worksharing loops: parallel for.
  \item Clauses to change effects: num\_threads, etc.
  \item schedule(static) important to introduce as will need it later.
  \item collapse() clause.
  \item Keep simple enough so reader knows notation, but refer to Common Core and Using OpenMP book for more details.
\end{itemize}

\subsection{Shared memory between threads}
\subsubsection{The OpenMP memory model}
\begin{itemize}
  \item What is a relaxed shared memory model
  \item Synchronization and the need to order operations
\end{itemize}
\subsubsection{Data sharing clauses}
\label{sssec:data_sharing}
\begin{itemize}
  \item private and firstprivate.
  \item shared and default(none) clauses.
  \item Need to be careful how we explain shared memory given the memory sharing rules between host/target we wi��ll introduce later.
\end{itemize}

\subsection{Tasks in OpenMP}
\begin{itemize}
   \item What is a task in OpenMP
   \item Explicit tasks 
   \item when do tasks complete and taskwait
   \item Asynchrony with tasks (task with nowait)
\end{itemize}

\section{Example: serial to parallel vector add with stack arrays}
\begin{itemize}
  \item vector add.
  \item arrays allocated on the stack instead of heap.
  \item no reduction yet.
  \item can reuse this as the first target example without needing map clauses.
  \item stack arrays might also be useful for first target example to highlight data sharing clauses.
\end{itemize}

\section{Reductions}
\label{sec:reduction}
\subsection{Example: serial to parallel pi}
\begin{itemize}
  \item Pi reduction example.
  \item Shows use of reduction clause.
\end{itemize}

\section{Our final goal}
Show the pragmas (BUD) needed to run a parallel program on a GPU.
This sets out the climax of the book, and excites that this book is going to explain it.


Introduce Matrix Multiply and
\Code{#pragma target teams distribute parallel for simd collapse(2)}
NB: put the arrays on the stack.

\section{another idea}

You probably already know quite a bit about OpenMP.  It is around 25 years old and shows no sign of fading away.  As long as
CPUs contain multiple cores, there will be need for OpenMP.   Anyone picking up this book has probably been exposed to 
OpenMP already.  If not, we recommend the book \emph{The OpenMP Common Core: making OpenMP simple again}.  
To make this book self contained, we will briefly cover the major topics in OpenMP that you need to understand in order to
work with a GPU using OpenMP.

\subsection{The key patterns}

Most programmers think of OpenMP as a system for turning serial loops into parallel loops.  Start with a well tested serial program.  
Then go through the following steps loop by loop to create a parallel program.
\begin{itemize}
\item Find a compute intensive loop.
\item Find the concurrency in the loop.  
\item Transform the loop body to expose the concurrency; that is, make changed needed 
so the loop iterations can execute in any order and the result will still be correct.
\item Add OpenMP directives to run the concurrent iterations in parallel
\end{itemize}
We show and example of this process in figure-pi-program

If parallel loops were all you could do with OpenMP, the language would not have lasted as long as it has. 



