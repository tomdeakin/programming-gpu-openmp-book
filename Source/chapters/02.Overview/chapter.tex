%%%%%%%%%%%%%%%%%%%%%%%%%%%%%%%%%%%%%%
%%  
%% Chapter 2: OpenMP overview
%%
%%%%%%%%%%%%%%%%%%%%%%%%%%%%%%%%%%%%%%

\def\ArtDir{02.Overview/figures}%

\chapter[OpenMP overview]{OpenMP overview}
\label{chapter:overview}

Note: Use C for examples in the book. Provide Fortran supplement.

\section{Set the Background/Context for the book.}

Starts with the {\bf problem} of trying to program a GPU.
Overview will shows everything a reader might need to review before the start on this book.

Will teach OpenMP 5.0. There are few relevant changes from 4.5, and we will call out any differences.

\section{Quick crash course in OpenMP}
This section will give the reader a very concise introduction into enough OpenMP for us to build on to explain the target directives to program GPUs.
\begin{itemize}
  \item What OpenMP shared memory is: cores share memory.
  \item Annotate code with compiler directives.
  \item API calls.
\end{itemize}

\subsection{Fork-join parallelism}
\label{ssec:fork_join}
\begin{itemize}
  \item Parallel: threads in a team.
  \item Worksharing loops: parallel for.
  \item Clauses to change effects: num\_threads, etc.
  \item schedule(static) important to introduce as will need it later.
  \item collapse() clause.
  \item Keep simple enough so reader knows notation, but refer to Common Core and Using OpenMP book for more details.
\end{itemize}

\subsection{Shared memory between threads}
\subsubsection{The OpenMP memory model}
\begin{itemize}
  \item What is a relaxed shared memory model
  \item Synchronization and the need to order operations
\end{itemize}
\subsubsection{Data sharing clauses}
\label{sssec:data_sharing}
\begin{itemize}
  \item private and firstprivate.
  \item shared and default(none) clauses.
  \item Need to be careful how we explain “shared memory” given the memory sharing rules between host/target we’ll introduce later.
\end{itemize}

\subsection{Tasks in OpenMP}
% Question do we do this now, or in the asynchronous chapter?
\begin{itemize}
   \item What is a task in OpenMP
   \item Explicit tasks 
   \item when do tasks complete and taskwait
   \item Asynchrony with tasks (task with nowait)
\end{itemize}

\section{Example: serial to parallel vector add with stack arrays}
\begin{itemize}
  \item vector add.
  \item arrays allocated on the stack instead of heap.
  \item no reduction yet.
  \item can reuse this as the first target example without needing map clauses.
  \item stack arrays might also be useful for first target example to highlight data sharing clauses.
\end{itemize}

\section{Reductions}
\label{sec:reduction}
\subsection{Example: serial to parallel pi}
\begin{itemize}
  \item Pi reduction example.
  \item Shows use of reduction clause.
\end{itemize}

\section{Our final goal}
Show the pragmas (BUD) needed to run a parallel program on a GPU.
This sets out the climax of the book, and excites that this book is going to explain it.


Introduce Matrix Multiply and
\Code{#pragma target teams distribute parallel for simd collapse(2)}
NB: put the arrays on the stack.



