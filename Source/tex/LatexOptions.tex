%%%%%%%%%%%%%%%%%%%%%%%%%%%%%%%%%%%%%%
%%  
%% Formatting suggestions/options to consider
%%
%%%%%%%%%%%%%%%%%%%%%%%%%%%%%%%%%%%%%%


\section{Helen Experimenting Tables for C/C++ and Fortran here}

If we decide on one table combined for C/C++ and Fortran, then it is choice 4 
and choice 6.   It we do two separate tables, one for C/C++, and one for Fortran, then it is choice 3 and choice 4.  I prefer one table combined, i.e., choices 4 
and 6.

The examples here show a simple table with one row only for each language 
(as in choices 5 and 6), or with two rows for each language (as in choices 
3 and 4).

\subsection{Choice 1}

This choice uses one table, with C/C++ and Fortran side by side.  This is 
what's currently in Chapter 4 that Tim has in Table 4.1.

% Illustration 1
\begin{table}[h!]
\centering
\caption{Choice 1: The form of directives in C/C++ and Fortran.  The 
combination of a directive and a structured 
block is called a construct.}
\label{fig:choice1}
\begin{tabular}{l|l}
\hline
\emph{C/C++}  & \emph{Fortran}\\
\hline
Directive format \\
\Code{#pragma omp construct [clause[, clause ...]]}    & \Code{!\$OMP construct [clause], clause ... ]]} \\                               
\hline 
Examples: directive plus structured block    \\
\Code{#pragma omp parallel private(x)}    & \Code{!\$OMP parallel private(x)} \\  
\Code{\{}                                                    &     \\
                                                                  &      \\
\Code{  ...code executed by each thread}    &     \Code{  ...code executed by each thread}  \\                                                                
                                                                  &      \\
\Code{\}}                                                    &   \Code{!\$OMP end parallel} \\
\hline
\end{tabular}
\end{table}


\subsection{Choice 2}

This choice uses one table, with C/C++ on the top and Fortran on the bottom

\begin{table}[h!]
\centering
\caption{Choice 2: The form of directives in C/C++ and Fortran.  
The combination of a directive and a structured 
block is called a construct.}
\label{fig:choice2}
\begin{tabular}{l}
\hline
\emph{C/C++} \\
\hline
Directive format: \\
\Code{#pragma omp construct [clause[, clause ...]]}    \\                               
\hline 
Examples: \\
directive plus structured block    \\
\Code{#pragma omp parallel private(x)}    \\  
\Code{\{}                                                  \\
                                                                \\
\Code{  ...code executed by each thread}    \\                                                                
                                                         \\
\Code{\}}                                                    \\
\\
\hline

\hline
\emph{Fortran}\\
\hline
Directive format: \\
\Code{!\$OMP construct [clause], clause ... ]]} \\                               
\hline 
Examples: \\
directive plus structured block    \\ 
\Code{!\$OMP parallel private(x)} \\  
   \\       
\Code{  ...code executed by each thread}  \\                                                                
\\
\Code{!\$OMP end parallel} \\
\hline
 

\end{tabular}
\end{table}



\subsection{Choice 3}

This choice uses two tables, with C/C++ as one table and Fortran as another 
table.  This is similar to the MPI books.

\begin{table}[h!]
\centering
\caption{Choice 3, Table 1/2: The form of directives in C/C++.  The 
combination of a directive and a structured 
block is called a construct.}
\label{fig:choice3-1}
\begin{tabular}{|l|}
\hline
Directive format: \\
\Code{#pragma omp construct [clause[, clause ...]]}    \\                               
\hline 
Examples: \\
directive plus structured block    \\
\Code{#pragma omp parallel private(x)}    \\  
\Code{\{}                                                  \\
                                                                \\
\Code{  ...code executed by each thread}    \\                                                                
                                                         \\
\Code{\}}                                                    \\
\\
\hline

\end{tabular}
\end{table}


\begin{table}[h!]
\centering
\caption{Choice 3, Table 2/2: The form of directives in Fortran.  The 
combination of a directive and a structured 
block is called a construct.}
\label{fig:choice3-2}
\begin{tabular}{|l|}

\hline
Directive format: \\
\Code{!\$OMP construct [clause], clause ... ]]} \\                               
\hline 
Examples: \\
directive plus structured block    \\ 
\Code{!\$OMP parallel private(x)} \\  
   \\       
\Code{  ...code executed by each thread}  \\                                                                
\\
\Code{!\$OMP end parallel} \\
\hline
 

\end{tabular}
\end{table}



\subsection{Choice 4}

This choice uses one table and shows directive format and example of structured block.  And this table also has vertial boundaries. 


\begin{table}[!h]
\centering
\caption{Choice 4: The form of directives in C/C++ and Fortran.  The 
combination of a directive and a structured 
block is called a construct.}
\begin{tabular}{|l|@{}l@{}|}\hline
\textbf{C/C++}
&
\begin{tabular}{l}
Directive format: \\
\textbf{\#pragma omp construct} \textit{[clause[, clause ...]]} \\ 
\hspace{6mm} structured block \\
\hline
Examples: \\
directive plus structured block \\
\#pragma omp parallel private(x) \\
\{ \\
\\
\hspace{5mm} code executed by eath thread \\
\\
\} \\
\end{tabular}
\tabularnewline\hline
\textbf{Fortran} 
&
\begin{tabular}{l}
Directive format: \\
\textbf{!\$OMP construct} \textit{ [clause], clause ... ]]} \\
\hspace{5mm} structured block \\
\textbf{!\$OMP end construct} \\
\hline
Examples: \\
directive plus structured block: \\
!\$OMP parallel private (x) \\
\\
\hspace{5mm} code executed by each thread \\
\\
!\$OMP end parallel \\

\end{tabular}
\tabularnewline\hline
\end{tabular}
\end{table}



\subsection{Choice 5}

This choice uses two tables, with C/C++ as one table and Fortran as 
another table.  This is similar to the MPI books.

\begin{table}[h!]
\centering
\caption{Choice 5, Table 1/2: The form of directives in C/C++.  The 
combination of a directive and a structured 
block is called a construct.}
\label{fig:choice5-1}
\begin{tabular}{|l|}
\hline
\textbf{\#pragma omp construct} \textit{[clause[, clause ...]]}    \\                               
\hspace{5mm} structured block \\
\hline

\end{tabular}
\end{table}


\begin{table}[h!]
\centering
\caption{Choice 5, Table 2/2: The form of directives in Fortran.  The 
combination of a directive and a structured 
block is called a construct.}
\label{fig:choice5-2}
\begin{tabular}{|l|}
\hline
\textbf{!\$OMP construct} \textit{ [clause], clause ... ]]}   \\
\hspace{5mm} structured block \\
\textbf{!\$OMP end construct} \\                               
\hline
 

\end{tabular}
\end{table}




\subsection{Choice 6}

This choice uses one table and only shows directive format as shown in the
 Ref Guide. This will be the most scenarios we need in the book. (since in 
 most cases we do not need to show examples for structures block).  This is 
 similar to Ruud's book. 
And the table has vertical boundaries.


\begin{table}[h!]
\centering
%\caption{Choice 6: The form of directives in C/C++ and Fortran.  The 
% combination of a directive and a structured 
%block is called a construct.}
\label{fig:choice6}
\begin{tabular}{|l|l|} \hline
\textbf{C/C++} & \textbf{\#pragma omp construct} \textit{[clause[, clause ...]]} \\ & \hspace{5mm} structured block \\                 \hline
\textbf{Fortran} & \textbf{!\$OMP construct} \textit{ [clause], clause ... ]]}  \\ & \hspace{5mm} structured block   \\ &   \textbf{!\$OMP end construct} \\     
                   
\hline

\end{tabular}
\caption{Choice 6: The form of directives in C/C++ and Fortran.  The 
combination of a directive and a structured 
block is called a construct.}
\end{table}


\subsection{Choice 7}

This choice uses one table and only shows directive format as shown in the 
Ref Guide. This will be the most scenarios we need in the book. (since in 
most cases we do not need to show examples for structures block).  This is 
similar to Ruud's book. 
They have only Figures, no Tables. Figure caption appears below the figure. 
 Figure number and figure caption appears on the same line. And the table 
 has vertical boundaries.




\begin{figure}[h!]
\centering
\label{fig:choice7}
%\caption{Choice 6: The form of directives in C/C++ and Fortran.  The 
% combination of a directive and a structured 
%block is called a construct.}
\begin{tabular}{|l|} \hline
\textbf{\#pragma omp construct} \textit{[clause[, clause ...]]} \\ 
\hspace{5mm} structured block \\                 
\hline
\textbf{!\$OMP construct} \textit{ [clause], clause ... ]]}  \\ 
\hspace{5mm} structured block   \\ 
\textbf{!\$OMP end construct} \\     
                  
\hline

\end{tabular}
\caption{\textbf{The form of directives in C/C++ and Fortran} - The 
combination of a directive and a structured 
block is called a construct.}
\end{figure}
